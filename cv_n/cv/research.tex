\cvsection{Research Experience}

\begin{cventries}
\cventry
{Advisor: Dr. Maxwell Parsons}
{Computer Generated Holography for Creating Optical Tweezers}
{University of Washington}
{2024 - Now}
{{\begin{cvitems}
\item Implemented iterative phase reconstruction algorithms to generate spot arrays using a phase spatial light modulators, reaching 92\% simulated power efficiency.
\item Investigating alternative neural network phase reconstruction algorithms for improved power efficiency and trap depth.
\item Extending algorithm to 3D trap arrays through wavefront propagation techniques.
\end{cvitems}
}}

\cventry
{Advisor: Dr. Qun Li}
{Improving the Scalability of Neural Network Surface Code Decoders}
{William \& Mary}
{2023 - 2024}
{{\begin{cvitems}
\item Designed transformer and structured selective state space models to decode the rotated planar code, a type of quantum error correction code.
\item Implemented and trained the models using PyTorch to decode low distance rotated planar codes.
\item Scaled decoders to higher distance codes using state compression techniques.
\end{cvitems}
}}

\cventry
{Advisor: Dr. Qun Li}
{Applying Differential Learning to Quantum Federated Learning}
{William \& Mary}
{2023}
{{\begin{cvitems}
\item Trained a federated QCNN using the Qiskit Machine Learning library, achieving 89\% simulator test accuracy and 70\% IBM QPU test accuracy on the MNIST dataset.
\item Implemented differential privacy to obfuscate sensitive client data, and performed a hyperparameter search to find an appropriate level of privacy. 
\end{cvitems}
}}

\cventry
{Advisor: Dr. Qun Li}
{First AI/ML Challenge at Dahlgren}
{NSWCDD}
{2022 - 2023}
{{\begin{cvitems}
\item Contributed to a white paper detailing relevant literature and proposed approaches on the weapon target assignment problem, which resulted in the team's acceptance to the competition.
\item Played a leading role in brainstorming and implementing approaches for automatic scheduling and coordination of advanced weapon systems.
\item Architected, implemented, and trained several approaches to reduce damage to high value assets, including a Deep Q-Learning agent and heuristic-driven Greedy agent.
\item The W\&M team won 3rd place and \$20,000 in prize money. 
\end{cvitems}
}}

\cventry
{Advisor: Dr. Chi-Kwong Li}
{Quantum Operator Approximation via Nonconvex PSD Programming}
{William \& Mary}
{2022}
{{\begin{cvitems}
\item Approximated arbitrary quantum operators using the Pauli product rotations, exponentiated elements of the Pauli group. 
\item Transformed problem into nonconvex positive semidefinite programming problem, and optimized using a trust-region approach.
\end{cvitems}
}}


\end{cventries}
