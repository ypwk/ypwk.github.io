\cvsection{Research Experience}

\begin{cventries}
\cventry
{Advisor: Dr. Maxwell Parsons}
{Generation of 3D Reconfigurable Holograms for Optical Control}
{University of Washington}
{2024-2025}
{{\begin{cvitems}
\item Designed and generated multi-depth 3D holographic phase patterns for experimental optical systems.
\item Studied tradeoffs between hologram dimensionality, control fidelity, and computational complexity.
\item Experimentally evaluated 3D holograms in laboratory optical setups.
\end{cvitems}
}}

\cventry
{Advisor: Dr. Maxwell Parsons}
{Phase Retrieval via Wirtinger-Flow for 3D Holographic Field Generation}
{University of Washington}
{2024-2025}
{{\begin{cvitems}
\item Investigated Wirtinger Flow-based optimization methods for phase retrieval in holographic field generation, designing new loss functions resulting in better performance than the current SoTA.
\item Studied convergence behavior and robustness of nonconvex optimization methods for multi-plane 3D holography.
\item Evaluated suitability of Wirtinger Flow methods for experimental holographic control using a phase-only spatial light modulator.
\end{cvitems}
}}

\cventry
{Advisor: Dr. Maxwell Parsons}
{Investigation of 3D Geometries for qLDPC Code Implementation}
{University of Washington}
{2024-2025}
{{\begin{cvitems}
\item Found that 3D layouts in neutral atom quantum computers have the potential to accelerate stabilizer measurement rounds for qLDPC codes by a significant factor, under certain assumptions.
\item Studied connectivity, routing, and locality constraints in 3D versus 2D hardware geometries.
\item Analyzed implications for scalable fault-tolerant quantum architectures.
\end{cvitems}
}}

\cventry
{Advisor: Dr. Maxwell Parsons}
{Construction and Characterization of a 2D Magneto-Optical Trap}
{University of Washington}
{2025}
{{\begin{cvitems}
\item Contributed to the assembly, alignment, and characterization of a 2D magneto-optical trap (MOT).
\item Assisted with optical alignment, magnetic field configuration, and system debugging.
\item Supported characterization of atomic beam flux and trap stability for downstream experimental use.
\end{cvitems}
}}

\cventry
{Advisor: Dr. Qun Li}
{Improving the Scalability of Neural Network Surface Code Decoders}
{William \& Mary}
{2023 - 2024}
{{\begin{cvitems}
\item Designed transformer and structured selective state space models to decode the rotated planar code, a type of quantum error correction code.
\item Implemented and trained the models using PyTorch to decode low distance rotated planar codes.
\item Scaled decoders to higher distance codes using state compression techniques.
\end{cvitems}
}}

\cventry
{Advisor: Dr. Qun Li}
{Applying Differential Learning to Quantum Federated Learning}
{William \& Mary}
{2023}
{{\begin{cvitems}
\item Trained a federated QCNN using the Qiskit Machine Learning library, achieving 89\% simulator test accuracy and 70\% IBM QPU test accuracy on the MNIST dataset.
\item Implemented differential privacy to obfuscate sensitive client data, and performed a hyperparameter search to find an appropriate level of privacy. 
\end{cvitems}
}}

\cventry
{Advisor: Dr. Qun Li}
{First AI/ML Challenge at Dahlgren}
{NSWCDD}
{2022 - 2023}
{{\begin{cvitems}
\item Contributed to a white paper detailing relevant literature and proposed approaches on the weapon target assignment problem, which resulted in the team's acceptance to the competition.
\item Played a leading role in brainstorming and implementing approaches for automatic scheduling and coordination of advanced weapon systems.
\item Architected, implemented, and trained several approaches to reduce damage to high value assets, including a Deep Q-Learning agent and heuristic-driven Greedy agent.
\item The W\&M team won 3rd place and \$20,000 in prize money. 
\end{cvitems}
}}

\cventry
{Advisor: Dr. Chi-Kwong Li}
{Quantum Operator Approximation via Nonconvex PSD Programming}
{William \& Mary}
{2022}
{{\begin{cvitems}
\item Approximated arbitrary quantum operators using the Pauli product rotations, exponentiated elements of the Pauli group. 
\item Transformed problem into nonconvex positive semidefinite programming problem, and optimized using a trust-region approach.
\end{cvitems}
}}


\end{cventries}
